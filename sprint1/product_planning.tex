\documentclass[11pt]{article}
% From the requirements: Font size to be used is Arial/11pt.
\usepackage{fontspec}
\usepackage{hyperref}
\setmainfont{Arial}
\usepackage{glossaries}

\makeglossaries

\newglossaryentry{Unity}
{
    name=Unity,
    description={Unity is a game development platform, can be used to make 3D environments. \url{http://unity3d.com/unity}}
}

\newglossaryentry{CleVR}
{
    name=CleVR,
    description={VR development team at Yes!Delft, focused on virtual reality therapy solutions. \url{http://clevr.net/}}
}

\newglossaryentry{MoSCoW}
{
    name=MoSCoW,
    description={Must have, Should have, Could have, and Would like but won't get. Clegg, Dai; Barker, Richard (2004-11-09). Case Method Fast-Track: A RAD Approach. Addison-Wesley.}
} 

\begin{document}

\title{HiVR Product Planning (Draft)}
\author{
	Adriaan de Vos - addevos - 4422643\\
	Boris Schrijver - brschrijver - 4315332\\
	Carlos Brunal - cbrunal - 4002725\\
	Leon Hoek - ljhoek - 4021606\\
	Wim de With - wdewith - 4295277
}
\date{\today}

\maketitle

\newpage
\tableofcontents
\newpage

\section{Introduction}
In this document, the following will be discussed: first, a MoSCoW model to determine which features are needed. Second, a product backlog with global user stories based on the features that are needed. Third, a release plan according to what should be finished following the road map. Last, a description of when a feature is done, when a sprint is done and when a release is done.

\section{Product}
	\subsection{High level product backlog}
    The \gls{MoSCoW} model for the project is:\\[\baselineskip]
    \begin{tabular}{ l l }
		Must have
            & Connect with \gls{Unity} VR environment running on another PC \\
        	& Generate map from Unity VR environment \\
            & Display static elements of the Unity VR environment \\
            & Display dynamic elements of the Unity VR environment \\
            & No performance impact on the Unity VR environment \\
            & Smooth movement on the map \\
            & Show subsection of the map located around the user \\
    	Should have
        	& Change the characteristics of characters in the Unity VR environment \\
        	& Move characters in the Unity VR environment \\
        	& Change TV screen in Unity VR environment \\
            & Play sounds in Unity VR environment \\
            & Small map displaying the entire Unity VR environment \\
    	Could have
        	& Touch screen support for the GUI \\
        	& Create scenarios in advance \\
            & Zooming in the map \\
            & Display trajectories of moving characters \\
            & An option to release the focus from the patient and freely move around on the map \\
            & logbook of commands given to characters for therapist \\
    	Won't have
        	& Realistic graphics \\
        	& Recording and playing back the events on the map \\
            & Connecting with more than one Unity VR environment at the same time \\
            & Connecting more than one client to the Unity VR environment at the same time \\
	\end{tabular}
    \subsection{Road map}
    \begin{tabular}{ l l }
		Sprint & Task \\
        \hline
        1 & Writing various design documents \\
        2 & Simple UI to display map \\
          & Initial protocol specifications \\
          & Finalizing design documents \\
        3 & Finalize protocol specification \\
          & Connecting to Unity from the GUI \\
          & Rendering a map with static elements in the GUI \\
        4 & Implementing the protocol in the Unity plugin \\
          & Implementing the protocol in the GUI program \\
        5 & Rendering dynamic elements on the map \\
          & Interacting with elements on the map \\
        6 & Displaying a small map of the entire environment \\
          & Touch screen support \\
        7 & Finalize project have bug free working build\\
        8 & Software launch and final presentation/report\\
	\end{tabular}
\section{Product Backlog}
	\subsection{User stories of features}
    
    \begin{itemize}
		\item As a therapist I want to see a clear visual overview of the virtual world
        \item As a therapist I want to be able to see the objects in the virtual world
        \item As a therapist I want to be able to see the moving objects in the virtual world
        \item As a therapist I want to be able to see trajectories of moving objects in the virtual world
        \item As a therapist I want to be able to see the patients location in the virtual world
        \item As a therapist I want to be able to see the patients field of view in the virtual world
        \item As a therapist I want to be able to see the state of objects in the virtual world
        \item As a therapist I want to be able to manipulate the state of objects in the virtual world
        \item As a therapist I want to be able to see which objects can be manipulated
        \item As a therapist I want to be able to touch objects to manipulate them
        \item As a therapist I want to change what the patient can see on the television
        \item As a therapist I want to play sounds in the virtual world
        \item As a therapist I want to play sounds from a specified location in the virtual world
        \item As a therapist I want to see a small mini-map displaying the entire map without the small details
        \item As a therapist I want to see the trajectories that actors will move after instructed to

	\end{itemize}
    
    \subsection{Initial release plan}
    The releases are planned from week 2 of quarter 4 which starts at April 18 till week 9 of quarter 4 which starts at June 14.\\[\baselineskip]
    \begin{tabular}{ l l }
		Week & Task \\
        \hline
        1 & Documentation and planning \\
        2 & Simple UI to display map \\
          & Design documents \\
        3 & Communication protocol specification \\
          & Unity plugin that can be connected with \\
          & Render a map with static elements in the UI \\
        4 & Protocol implementation in both the UI and the Unity plugin \\
        5 & Render dynamic elements on the map \\
          & Option to interact with elements on the map \\
        6 & A small map displaying the entire environment \\
          & Touch screen support \\
        7 & Feature complete final release \\
        8 & Feature complete bug free release and final report \\
	\end{tabular}

\section{Definition of done}

	\subsection{Backlog items}
	A feature is done when the following points are met:
	\begin{itemize}
		\item All criteria of the corresponding user story are met
		\item Continuous integration build is successful
		\item 80\% or higher test coverage
	    \item Can be demonstrated to other team members without problems
		\item Approved by two other team members after manual functional test
	    \item Can be merged into current master
	    \item Code is documented and no Visual Studio extensions warnings
	\end{itemize}

	\subsection{Sprint}
	A sprint is done when all items in the backlog are done according to the preceding definition and no critical bugs exist in the issue tracker.

	\subsection{Release}
	A release is done when \gls{CleVR} gets the demo and documentation.

\section{Glossary}
\printglossary[title=]

\end{document}